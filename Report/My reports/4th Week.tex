%%%%%%%%%%%%%%%%%%%%%%%%%%%%%%%%%%%%%%%%%
% University/School Laboratory Report
% LaTeX Template
% Version 3.1 (25/3/14)
%
% This template has been downloaded from:
% http://www.LaTeXTemplates.com
%
% Original author:
% Linux and Unix Users Group at Virginia Tech Wiki 
% (https://vtluug.org/wiki/Example_LaTeX_chem_lab_report)
%
% License:
% CC BY-NC-SA 3.0 (http://creativecommons.org/licenses/by-nc-sa/3.0/)
%
%%%%%%%%%%%%%%%%%%%%%%%%%%%%%%%%%%%%%%%%%

%----------------------------------------------------------------------------------------
%	PACKAGES AND DOCUMENT CONFIGURATIONS
%----------------------------------------------------------------------------------------

\documentclass{article}

\usepackage[version=3]{mhchem} % Package for chemical equation typesetting
\usepackage{siunitx} % Provides the \SI{}{} and \si{} command for typesetting SI units
\usepackage{graphicx} % Required for the inclusion of images
\usepackage{natbib} % Required to change bibliography style to APA
\usepackage{amsmath} % Required for some math elements 
\usepackage{bm}
\usepackage{amssymb}
\setlength\parindent{0pt} % Removes all indentation from paragraphs

\renewcommand{\labelenumi}{\alph{enumi}.} % Make numbering in the enumerate environment by letter rather than number (e.g. section 6)

\usepackage{times} % Uncomment to use the Times New Roman font
\usepackage{verbatim}
\usepackage{tabulary}
\usepackage{algorithmic}
\usepackage{algorithm}
%----------------------------------------------------------------------------------------
%	DOCUMENT INFORMATION
%----------------------------------------------------------------------------------------

\title{NYU-MMVC-LAB Weekly Report} % Title

\author{\textsc{Cheng Qian}} % Author name

\date{\today} % Date for the report
%\date{September 13, 2015}

\begin{document}

\maketitle % Insert the title, author and date


% If you wish to include an abstract, uncomment the lines below
% \begin{abstract}
% Abstract text
% \end{abstract}

%----------------------------------------------------------------------------------------
%	SECTION 1
%----------------------------------------------------------------------------------------

\section{Motivation}

To fully understand the methods and technologies behind the TextBox detection (Liao et al. 2017) and replicate its results in Python TensorFlow environment, and then proceed with other researches in computer vision areas.

%----------------------------------------------------------------------------------------
%	SECTION 2
%----------------------------------------------------------------------------------------

%\section{Methods}

%----------------------------------------------------------------------------------------
%	SECTION 3
%----------------------------------------------------------------------------------------

\section{Reading}

This week I've primarily focused on Tensorflow study, the reading materials covered
\begin{itemize}
	\item MNIST application with simple softmax classifier
	\item MNIST appliaction with convolutional neural network classifier
	\item Iris Data with deep neural network classifier
	\item Loop training, graph construction and feeding
	\item Sharing variables, threading and queues
\end{itemize}

Aside from that, I've also read the TextBox paper to familiarize myself with the topic, outlining the key components of the model structure for next week's study.
\begin{itemize}
	\item Based on a text recognition algorithm CRNN (Shi, Bai, and Yao, 2015) which estimates sequence probability conditioned on input image
	\item Word-based text detection with VGG-16 architecture (Simonyan and Zisserman 2014)
	\item Multiple output layers (called \textit{text-box layers}) are inserted after the last and some intermediate convolutional layers, simultaneously predicting text presence and bounding boxes
	\item Non-maximum suppression (NMS) process
\end{itemize}



%----------------------------------------------------------------------------------------
%	BIBLIOGRAPHY
%----------------------------------------------------------------------------------------
%----------------------------------------------------------------------------------------

\end{document}
